\documentclass[11pt]{beamer} \usetheme{Frankfurt}

\usepackage[utf8]{inputenc} \usepackage[T1]{fontenc}
\usepackage{textpos}
\usepackage{booktabs}
\usepackage{adjustbox}
\colorlet{beamer@blendedblue}{teal!40!teal}
\def\sym#1{\ifmmode^{#1}\else\(^{#1}\)\fi}

\author{Manuel Buchmann}

\title[Elasticities]{Estimating Sector-Specific Elasticities of
  Substitution in the Swiss Labor Market}

\date[\today]{Internal Presentation\\25.01.2019}
%\logo{\includegraphics[scale=0.5]{Z:/Logos/UniBas_Logo_EN_Schwarz_RGB_65}}

\setbeamercovered{transparent}

\setbeamertemplate{navigation symbols}{}


\begin{document}
\maketitle
\section{Introduction}
\subsection{Motivation}
\begin{frame}
  \frametitle{Motivation}
  \begin{itemize}
  \item I am building a macro model of Switzerland with an important
    labor market component
  \item For this model I need realistic, industry-specific estimates
    of the elasticity of substitution between low-skill and high-skill
    labor
  \item However, almost all existing research focuses on the USA
    \begin{itemize}
    \item College graduates are considered high-skill and high school
      graduates are considered low-skill
    \item Due to the dual education system, this distinction makes
      much less sense in Switzerland, need to find an alternative
    \item Proposal: Use occupation-based classification (Competence
      Levels)
    \end{itemize}
  \end{itemize}
\end{frame}

\subsection{Research Questions}
\begin{frame}
  \frametitle{Research Questions}
  \begin{block}{Research Question 1}
    Are Competence Levels a better tool for measuring the skill level
    of a worker than education levels?
  \end{block}
  \begin{block}{Research Question 2}
    If so, what are the implied industry-specific elasticities of
    substitution between skill levels in Switzerland?
  \end{block}
\end{frame}
% \subsection{Evidence}
% \begin{frame}
%   \frametitle{Evidence}
%   \begin{figure}[ht]
%     \centering
%     \includegraphics[width=0.8\textwidth]{Z:/OLG_CGE_Model/code/elasticities/coll_tert.png}
%     \caption{Share of college and tertiary degree graduates}
%     \label{fig1}
%   \end{figure}
% \end{frame}

\begin{frame}
  \frametitle{What are Competence Levels?}
  % Table generated by Excel2LaTeX from sheet 'Sheet1'
  \begin{table}[htbp]
    \centering
    \begin{tabular}{rp{16.645em}}
      \toprule
      \multicolumn{1}{l}{Competence Level} & \multicolumn{1}{l}{Description} \\
      \midrule
      1     &  Tasks with complex problem solving and decision making which require a large amount of factual and theoretical knowledge in an area of expertise \\
      2     & Complex practical tasks requiring a large amount of knowledge in an area of expertise \\
      \midrule
      3     & Practical tasks like retail, care, data processing and administration, operation of machines and eletronical equipment, security service or driving \\
      4     & Simple manual tasks \\
      \bottomrule
    \end{tabular}%
    \label{tab:addlabel}%
  \end{table}%
\end{frame}

\begin{frame}
  \frametitle{Why are education levels not a suitable proxy for skill
    in CH?}
  \begin{itemize}
  \item Due to the dual education system, education for many
    occupations is much more applied
    \begin{itemize}
    \item Many highly skilled workers never complete university or any
      tertiary education
    \item Instead they complete an apprenticeship and later more
      specialized training which may or may not be recognized as a
      tertiary education
    \item Many of these occupations are highly regarded, very
      skill-intensive and well-paid
    \end{itemize}
  \item With the creation of technical colleges in the mid-nineties
    the share of labor force with a higher education started to
    increase drastically, making comparisons over the last 30 years
    even harder
  \item For these reasons it may be more suitable to categorize the
    Swiss labor force based on actual occupations
  \end{itemize}
\end{frame}

\begin{frame}
  \frametitle{Preview of Results}
  \begin{table}[htbp]
    \centering
    \begin{tabular}{ccc}
      \toprule
      \multicolumn{1}{l}{Elasticities} & \multicolumn{1}{l}{Education-Based}
      & \multicolumn{1}{l}{Occupation-Based} \\
      \midrule
      CH    &  -18.01   & 2.037\sym{***} \\
      USA   & 1.996\sym{***}  & -2.846\sym{***} \\
      \bottomrule
    \end{tabular}%
  \end{table}%
  Education-Based elasticities of substitution for Switzerland are
  statistically insignificant (suggesting an elasticity of infinity)
  and negative (suggesting an incompatibility with the theoretical
  framework). Occupation-Based elasticities of substitution in the
  United States are clearly incompatible with the theoretical
  framework. The other elasticities are very comparable, plausible and
  compatible with theory and previous research.
\end{frame}
  
\subsection{Literature}
\begin{frame}
  \frametitle{Literature}
  \begin{itemize}
  \item Basic Model: Katz and Murphy (1992)
  \item Extensions: Card and Lemieux (2001) (Age Groups), Blankenau
    and Cassou (2011) (Industries)
  \item European Context: Fitzenberger and Kohn (2006)
  \item Immigration: Borjas, Grogger and Hanson (2008, 2010), Aydemir
    and Borjas (2007), Ottaviano and Peri (2011), ...
  \item Occupations as skill levels: Orrenius and Zavodny (2007)
  \end{itemize}
\end{frame}
\section{Theory and Estimation}
\subsection{Theory}
\begin{frame}
  \frametitle{Theory (1)}
  \begin{block}{Assumption 1}
    The labor module of the aggregate production function is a CES
    function consisting of unskilled and skilled labor.
    \begin{equation}
      \label{eq:1}
      Y_{t} = f(A_{t},K_{t},L_{t})
    \end{equation}
    and
    \begin{equation}
      \label{eq:2}
      L_{t}=\left( \theta_{u,t}U_{t}^{\rho} + \theta_{s,t}S_{t}^{\rho} \right)^{\frac{1}{\rho}}
    \end{equation}
  \end{block}
  \begin{block}{Research Question 1}
    Which definition should we use for $U$ and $S$?
  \end{block}
  \begin{block}{Research Question 2}
    What is the value of $\frac{1}{1-\rho}$?
  \end{block}
\end{frame}

\begin{frame}
  \frametitle{Theory (2)}
  \begin{block}{Assumption 2}
    Firms maximize profits and markets are competitive.
  \end{block}
  \begin{align}
    \label{eq:3}
    w_{u,t}&=\frac{\partial Y_{t}}{\partial U_{t}}=\frac{d
             Y_{t}}{d L_{t}}\frac{d L_{t}}{d U_{t}} \\
           & = \frac{d Y_{t}}{d L_{t}} \left(
             \theta_{u,t}U_{t}^{\rho} + \theta_{s,t}S_{t}^{\rho}
             \right)^{\frac{1}{\rho}-1}  \theta_{u,t}
             U_{t}^{\rho-1} \\
    w_{s,t} &= \frac{d Y_{t}}{d L_{t}} \left(
              \theta_{u,t}U_{t}^{\rho} + \theta_{s,t}S_{t}^{\rho}
              \right)^{\frac{1}{\rho}-1} 
              \theta_{s,t}S_{t}^{\rho-1} \\
    \frac{w_{s,t}}{w_{u,t}}&= \frac{\theta_{s,t}}{\theta_{u,t}}\left( \frac{S_{t}}{U_{t}} \right)^{\rho-1}
  \end{align}
\end{frame}
\subsection{Estimation}
\begin{frame}
  \frametitle{Estimation}
  Let $\tilde{w_{t}}\equiv \frac{w_{s,t}}{w_{u,t}}$,
  $\tilde{\theta_{t}}\equiv \frac{\theta_{s,t}}{\theta_{u,t}}$, and
  $\tilde{s_{t}}\equiv \frac{S_{t}}{U_{t}}$. Then taking logs yields
  \begin{equation}
    \label{eq:4}
    \log(\tilde{w_{t}}) = \log(\tilde{\theta_{t}}) + (\rho-1)\log(\tilde{s_{t}})
  \end{equation}
  Additionally, substitute in the elasticity of substitution
  $\sigma \equiv \frac{1}{1-\rho}$:
  \begin{equation}
    \label{eq:5}
    \log(\tilde{w_{t}}) = \log(\tilde{\theta_{t}}) - \frac{1}{\sigma}\log(\tilde{s_{t}})
  \end{equation}
  This equation can be estimated as
  \begin{equation}
    \label{eq:6}
    log(\tilde{w_{t}})=\beta_{0}+\beta_{1}t+\beta_{2}log(\tilde{s_{t}})+\varepsilon_{t}
  \end{equation}
  where $\beta_{1}$ measures the growth rate of the relative skill
  share parameter (skill-biased technological change) and
  $\beta_{2}=-\frac{1}{\sigma}$
  \begin{block}{Assumption 3}
    Within a period, skill shares drive wage gaps. Wage Gaps do not
    affect skill shares.
  \end{block}
\end{frame}

% \subsection{Intuition}
% \begin{frame}
%   \frametitle{ Intuition}
%   \begin{block}{Hypothesis}
%     The elasticity of substitution is always positive or
%     zero. Therefore, controlling for technological change, I expect a negative relationship between
%     $\tilde{s_{t}}$ and $\tilde{w_{t}}$.
%   \end{block}
%   \begin{itemize}
%   \item If there is more high skill labor, high skill wages should go
%     down
%   \item If it is easy to substitute low and high skill labor, this
%     effect should be small
%     \begin{itemize}
%     \item Because: if low skill can easily replace high skill, then
%       high skill was never all that scarce to begin with
%     \end{itemize}
%   \item If it is hard to substitute low and high skill labor, this
%     effect should be big
%     \begin{itemize}
%     \item Because: high skill was really scarce before, so them
%       becoming more abundant is making a big difference
%     \end{itemize}
%   \end{itemize}
  
% \end{frame}

% \section{Age-Specific Effects}
% \subsection{Theory}
% \begin{frame}
%   \frametitle{Theory}
%   \begin{block}{Assumption 4}
%     Workers within a skill group but in different age groups are
%     imperfect CES substitutes.
%     \begin{equation}
%       \label{eq:7}
%       U_{t}=\left[
%         \sum_{j}\alpha_{j}U_{j,t}^{\eta}\right]^{\frac{1}{\eta}}
%       \text{and } S_{t}=\left[  \sum_{j}\beta_{j}S_{j,t}^{\eta}\right]^{\frac{1}{\eta}} 
%     \end{equation}
%   \end{block}
%   Integrating these changes yields the following equation:
%   \begin{equation}
%     \label{eq:8}
%     \frac{w_{s,j,t}}{w_{u,j,t}}= \frac{\theta_{s,t}}{\theta_{u,t}}\left(
%       \frac{S_{t}}{U_{t}} \right)^{\rho-\textcolor{red}{\eta}} \textcolor{red}{\frac{\beta_{j}}{\alpha_{j}}\left( \frac{S_{j,t}}{U_{j,t}} \right)^{\eta-1}}
%   \end{equation}
% \end{frame}
% \subsection{Estimation}
% \begin{frame}
%   \frametitle{Estimation}
%   After some reformulating, we arrive at the following equation:
%   \begin{align}
%     \label{eq:9}
%     \log(\tilde{w_{j,t}}) = \log(\tilde{\theta_{t}})
%     \textcolor{red}{+log\left( \frac{\beta_{j}}{\alpha_{j}} \right)} -
%     \frac{1}{\sigma_{E}}\log(\tilde{s_{t}}) \\\textcolor{red}{-\frac{1}{\sigma_{A}}\left(  \log\left( \tilde{s_{j,t}}
%     \right)-\log \left( \tilde{s_{t}} \right) \right)}
%   \end{align}
%   \begin{itemize}
%   \item $\frac{\beta_{j}}{\alpha_{j}}$ is captured with age fixed effects
%   \item Problem: $\tilde{s_{t}}$ depends on $\sigma_{A}$, we cannot disentangle
%     the two terms
%   \item Solution: Two-Step Procedure
%   \end{itemize}
% \end{frame}

% \begin{frame}
%   \frametitle{Estimation (2)}
%   \begin{itemize}
%   \item (1): Estimate $\sigma_{A}$ by controlling for a joint
%     technology and aggregate supply effect as follows
%     \begin{equation}
%       \label{eq:10}
%       \log(\tilde{w_{j,t}})=J_{j}+T_{t}-\frac{1}{\sigma_{A}}\log(\tilde{s_{j,t}})+e_{j,t}
%     \end{equation}
%     Using $\widehat{\sigma_{A}}$, one can estimate the age-group
%     efficiency parameters $\widehat{\alpha_{j}}$ and
%     $\widehat{\beta_{j}}$, which in turn can be used to calculate the
%     aggregate supply ratio $\widehat{\tilde{s_{t}}}$ 
%   \end{itemize}
% \end{frame}

% \begin{frame}
%   \frametitle{Estimation (3)}
%   \begin{itemize}
%   \item (2): Estimate $\sigma_{E}$ with the following equation using the estimated
%     $\widehat{\tilde{s_{t}}}$
%     \begin{align}
%       \label{eq:11}
%       \log(\tilde{w_{t}}) = \log(\tilde{\theta_{t}})
%     +log\left( \frac{\beta_{j}}{\alpha_{j}} \right) -
%     \frac{1}{\sigma_{E}}\log(\widehat{\tilde{s_{t}}}) \\ -\frac{1}{\sigma_{A}}\left(  \log\left( \tilde{s_{j,t}}
%     \right)-\log ( \widehat{\tilde{s_{t}}} ) \right)
%     \end{align}
%     \item This estimation also yields another estimate of $\sigma_{A}$
%       which should be similar to the previously estimated estimate.
%   \end{itemize}
% \end{frame}

% \subsection{Intuition}
% \begin{frame}
%   \frametitle{Intuition}
%   \begin{itemize}
%   \item If workers of different age are hard to substitute (low
%     elasticity of substitution), then a big
%     shift of this age-groups skill composition relative to the
%     aggregate skill composition will have a big effect on the wage ratio.  
%   \end{itemize}
% \end{frame}

%\section{Industry-Specific Effects}
\subsection{Industry-Specific Effects}
\begin{frame}
  \frametitle{Industry-Specific Effects}
  Implementing heterogeneity over industries is straight forward in
  theory, just add an industry index $i$
  \begin{equation}
    \label{eq:12}
    \log(\tilde{w_{\textcolor{red}{i,}t}}) = \log(\tilde{\theta_{\textcolor{red}{i,}t}}) - \frac{1}{\sigma_{\textcolor{red}{i}}}\log(\tilde{s_{\textcolor{red}{i,}t}})
  \end{equation}
\end{frame}
\subsection{Estimation - Industry}
\begin{frame}
  \frametitle{Estimation - Industry}
  \begin{itemize}
  \item Hoewever: Assumption 3 is violated!
  \item If labor is mobile between sectors, high skill labor will move
    to sectors with a high relative wage, causing endogeneity issues

  \item Solution: IV
  \end{itemize}

\end{frame}
\subsection{IV}
\begin{frame}
  \frametitle{Estimation: IV}
  \begin{itemize}
  \item There is a natural instrument: the aggregate skill ratio
  \item If assumption 3 holds for the aggregate economy, then the
    aggregate skill ratio is not driven by wages
  \item However, by construction it is correlated to the industry
    skill ratios
  \item Therefore, I estimate industry-specific elasticities of
    substitution with 2SLS, using the aggregate skill ratio as an
    instrument for the industry-specific skill ratio
  \end{itemize}
\end{frame}
\section{Data}
\subsection{Data}
\begin{frame}
  \frametitle{Data Sources}
  \begin{itemize}
  \item Switzerland
    \begin{itemize}
    \item Swiss Labor Force Survey
    \item 1992 - 2017, time period covers 2 years
    \end{itemize}
  \item USA
    \begin{itemize}
    \item Current Population Survey, March Extracts
    \item 1982 - 2016, time period covers 2 years
    \end{itemize}
  \end{itemize}
\end{frame}

\begin{frame}
  \frametitle{Sample Selection and Estimation of Wage Gaps}
  Sample Selection as in Katz and Murphy (1992):
  \begin{itemize}
  \item Sample for wage gaps: Only full time employed workers aged
    26-60
  \item Sample for skill share: All workers aged 15-65, including
    self-employed and part time
  \end{itemize}
  Estimation of shares:
  \begin{itemize}
  \item Wage gaps: Regression of hourly (weekly) log wages on skill level
    dummy, gender dummy, (non-white dummy) and linear age term in
    every time period and age group (and industry). Inverse of
    variance is later used as weight in the main regression.
  \item Skill shares: Ratio of sum of hours worked
  \end{itemize}
\end{frame}

\begin{frame}
  \frametitle{Education Skill Definitions }
  \begin{itemize}
  \item Switzerland
    \begin{itemize}
    \item High Skill: Has university degree (includes technical and
      pedagogical colleges)
    \item Low Skill: Completed apprenticeship
    \item Everything else is omitted (i.e. Gymnasium, higher
      vocational education, etc.)
    \end{itemize}
  \item USA
    \begin{itemize}
    \item High Skill: Completed College or ``Advanced''
    \item Low Skill: Completed High School
    \item Omitted: ``Some College'', ``Less than High School''
    \end{itemize}
  \end{itemize}
\end{frame}
\begin{frame}
  \frametitle{Data Issues}
  \begin{itemize}
  \item Competence Levels are based on ISCO-Classification which is
    European and thus not in US data directly: Have to rely on
    crosswalks, recoding of occupations
    not possible 1:1
    \begin{itemize}
    \item Example: ``\emph{Veterinary Assistants and Laboratory Animal
      Caretakers}'' in the US CPS data corresponds to either ``\emph{Veterinary
      technicians and assistants}'' (high skill) or ``\emph{Pet groomers and
      animal care workers}'' (low skill) in the ISCO
    \item This leads to some doubt about the validity of the use of
      Swiss competence levels in US data
    \end{itemize}
  \end{itemize}
\end{frame}

\section{Descriptive Evidence}
\begin{frame}
  \frametitle{Switzerland - Occupations}
  \begin{figure}[ht]
    \centering
    \includegraphics[width=0.8\textwidth]{Z:/OLG_CGE_Model/code/elasticities/7inds/aggr_shares_skill.png}
    \caption{Relative Wages and Employment - Switzerland, Occupation}
  \end{figure}
\end{frame}
\begin{frame}
  \frametitle{Switzerland - Education}
  \begin{figure}[ht]
    \centering
    \includegraphics[width=0.8\textwidth]{Z:/OLG_CGE_Model/code/elasticities/7inds/aggr_shares_coll.png}
    \caption{Relative Wages and Employment - Switzerland, Education}
  \end{figure}
\end{frame}
\begin{frame}
  \frametitle{USA - Occupations}
  \begin{figure}[ht]
    \centering
    \includegraphics[width=0.8\textwidth]{Z:/OLG_CGE_Model/code/elasticities/7inds/aggr_shares_skill_us.png}
    \caption{Relative Wages and Employment - USA, Occupation}
  \end{figure}
\end{frame}
\begin{frame}
  \frametitle{USA - Education}
  \begin{figure}[ht]
    \centering
    \includegraphics[width=0.8\textwidth]{Z:/OLG_CGE_Model/code/elasticities/7inds/aggr_shares_coll_us.png}
    \caption{Relative Wages and Employment - USA, Education}
  \end{figure}
\end{frame}
\begin{frame}
  \frametitle{Switzerland - Industries, Occupation}
  \begin{figure}[ht]
    \centering
    \includegraphics[width=0.8\textwidth]{Z:/OLG_CGE_Model/code/elasticities/7inds/laborwage.png}
    \caption{Relative Wages and Employment - Switzerland}
  \end{figure}
\end{frame}



\section{Results}
\subsection{Aggregate}

% \begin{frame}
%   \frametitle{Aggregate Economy Results - no Age Effects}
%   \centering \adjustbox{max height=\dimexpr\textheight-5.5cm\relax,
%     max width=\textwidth}{
%     \input{Z:/OLG_CGE_Model/code/elasticities/7inds/main_us.tex} }
% \end{frame}

\begin{frame}
  \frametitle{Aggregate Economy Results - no Age Effects}
  \centering \adjustbox{max height=\dimexpr\textheight-5.5cm\relax,
    max width=\textwidth}{
    \input{Z:/OLG_CGE_Model/code/elasticities/7inds/main_ch.tex} }
\end{frame}

% \begin{frame}
%   \frametitle{Aggregate Economy Results, Overview}
%   \begin{figure}[ht]
%     \centering
%     \includegraphics[width=0.8\textwidth]{Z:/OLG_CGE_Model/code/elasticities/7inds/mainresults.png}
%     \caption{Aggregate Results, Preferred Models}
%     \label{fig2}
%   \end{figure}
% \end{frame}
    

    
\begin{frame}
  \frametitle{Industry-Specific Results - no Age Effects}
  \centering \adjustbox{max height=\dimexpr\textheight-5.5cm\relax,
    max width=\textwidth}{

    \input{Z:/OLG_CGE_Model/code/elasticities/noage/ind_skill.tex}

  }
\end{frame}

% \begin{frame}
%   \frametitle{Industry-Specific Results - with Age Effects}
%   \centering \adjustbox{max height=\dimexpr\textheight-5.5cm\relax,
%     max width=\textwidth}{

%     \input{Z:/OLG_CGE_Model/code/elasticities/7inds/ind_skill.tex}

%   }
% \end{frame}

\begin{frame}
  \frametitle{Industry-Specific Results - Overview}
  \begin{figure}[ht]
    \centering
    \includegraphics[width=0.8\textwidth]{Z:/OLG_CGE_Model/code/elasticities/noage/indresults_all.png}
    \caption{Industry Results, Preferred Models}
    \label{fig3}
  \end{figure}
\end{frame}

\section{Conclusion}

\begin{frame}
  \frametitle{Main Take-Aways}
  \begin{itemize}
  \item Occupation as a proxy for skill level describes the Swiss
    labor market much better than Education. Do not blindly adopt
    definitions from USA 
  \item Elasticities of substitution vary between industries, but less
    than expected
  \item The Swiss finance sector is special in that respect
  \item Not all sectors' elasticities can be estimated from
    quantitative data, need to rely on qualitative data as well
  \end{itemize}
\end{frame}

\begin{frame}
  \frametitle{End of Presentation}
  Thank you for your attention!
\end{frame}

\end{document}