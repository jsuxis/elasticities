\documentclass[]{article}
\usepackage{authblk}
\usepackage{pdfpages}
\usepackage{comment}
\usepackage{booktabs,tabulary}
\usepackage{natbib}
\usepackage{tabu}
\usepackage{graphicx}
\usepackage{caption}
\usepackage{amssymb}
\usepackage{amsmath}
\usepackage{colortbl}
\usepackage{float}
\bibliographystyle{chicago}
\setcitestyle{authoryear, open={(},close={)}}
\graphicspath{{img/}}

%section{Estout related things}
% *****************************************************************
% Estout related things
% *****************************************************************
\newcommand{\sym}[1]{\rlap{#1}}% Thanks to David Carlisle

\let\estinput=\input% define a new input command so that we can still flatten the document

\newcommand{\estwide}[3]{
  \vspace{.75ex}{
    \begin{tabulary}
      {0.7\textwidth}{@{\hskip\tabcolsep\extracolsep\fill}l*{#2}{#3}}
      \toprule
      \estinput{#1}
      \bottomrule
      \addlinespace[.75ex]
    \end{tabulary}
  }
}	

\newcommand{\estauto}[3]{
  \vspace{.75ex}{
    \begin{tabular}{l*{#2}{#3}}
      \toprule
      \estinput{#1}
      \bottomrule
      \addlinespace[.75ex]
    \end{tabular}
  }
}

% Allow line breaks with \\ in specialcells
\newcommand{\specialcell}[2][c]{%
  \begin{tabular}[#1]{@{}c@{}}#2\end{tabular}}

% *****************************************************************
% Custom subcaptions
% *****************************************************************
% Note/Source/Text after Tables
\newcommand{\figtext}[1]{
  \vspace{-1.9ex}
  \captionsetup{justification=justified,font=footnotesize}
  \caption*{\hspace{6pt}\hangindent=1.5em #1}
}
\newcommand{\fignote}[1]{\figtext{\emph{Note:~}~#1}}

\newcommand{\figsource}[1]{\figtext{\emph{Source:~}~#1}}

% Add significance note with \starnote
\newcommand{\starnote}{\figtext{* $p < 0.1$, ** $p < 0.05$, *** $p < 0.01$. Standard errors in parentheses.}}

% *****************************************************************
% siunitx
% *****************************************************************
\usepackage{siunitx} % centering in tables
\sisetup{
  detect-mode,
  tight-spacing		= true,
  group-digits		= false ,
  input-signs		= ,
  input-symbols		= ( ) [ ] - + *,
  input-open-uncertainty	= ,
  input-close-uncertainty	= ,
  table-align-text-post	= false
}



%opening
\title{Estimating Sector-Specific Elasticities of Substitution in the
  Swiss Labor Market\footnote{Unfinished Draft. Do not quote or circulate.}}
\author{Manuel Buchmann}

\begin{document}

\maketitle

\begin{abstract}
  Most research on substitutability between high- and low skill labor
  uses data from the United States. Therefore, high- and low skill
  labor is usually defined as having attained a college or high school
  degree respectively. However, this distinction is much less sensible
  in a country like Switzerland, where, due to its dual education
  system, many high skill workers do not possess a traditional college
  degree or even a tertiary education. I suggest an alternative
  distinction between high- and low skill workers based on occupations
  and estimate their elasticitiy of substitution in the Swiss economy
  in general as well as in specific sectors using data from 1991 -
  2017. I then replicate the estimation with data from the United
  States and compare the results, concluding that an occupation-based
  classification of skills is much better suited for analyses in
  Switzerland, while an education-based classification still gives
  much more realistic results for the US. Methodologically, I use a
  \cite{Katz.1992} framework, combining the work of \cite{Card.2001}
  who allow for imperfect substitution of different age groups with
  the work of \cite{Blankenau.2011} who develop a method of estimating
  sector-specific elasticities of substitution.
\end{abstract}

\newpage
\section{Introduction}
\label{intro}
Information about the substitutability between workers of different
skill types gives us important insights into the production technology
in the aggregate economy and specific industries of a country. Such
information is crucial for realistic macroeconomic models but can also
be used for qualitative analyses. However, the existing literature on
this subject is highly influenced by the American system, where it is
a reasonable simplification to assume that a worker with a college
degree is a high-skill worker and a worker with a high school degree
is a low-skill worker, thus defining the skill premium as a college
premium. This distinction is less reasonable in countries like
Germany, Switzerland or Austria, where a dual education system is in
place. In such systems, education in many professions is much more
applied from a young age on. Often, pupils start an apprenticeship
directly after compulsory schooling starting around the age of
15. Apprenticeships usually include some general education and are
often (but not always) followed by further specialized training, which
in some cases can take a long time and can be very
skill-intensive. Apprentices also have the option of completing a
Federal Vocational Baccalaureate which which is an extended general
education program and, under some conditions, qualifies for attending
university. In practice, many (but not all) vocational training
degrees are well paid, highly regarded and should be considered
``high-skill''. Identification of these cases in usual data sources
however, is impossible to do consistently.

Another reason for why education is a bad proxy for skill in
Switzerland is the creation of the technical colleges
(``Fachhochschulen'') in 1995. These colleges are ranked on the same
level as universities, however they are much more applied and often
attended after completion of an apprenticeship and a so called
Vocational Baccalaureate. A worker who completed such an education
should clearly be considered as a high-skill worker. An older worker
who did not have the opportunity to attend such a college might have
the same actual skill level, however a researcher might label him as a
low skill worker because his education was less well defined. These
technical colleges have been extremely popular and led to a big
increase in the share of the population with a tertiary degree in
Switzerland since the year 2000.

Instead of using a college or other tertiary degrees as a classifier
for skill, I propose a classification based on occupations. More
specifically, I classify occupations based on the International
Standard Classification of Occupations version 2008 (ISCO-08) into
four competence levels. This classification is adopted from the Swiss
Earnings Structure Survey \citep{SwissStatisticalOffice.2016}. The
Swiss Federal Statistical Office defines the four competence level as
follows. Competence level 1 includes tasks with complex problem
solving and decision making which require a large amount of factual
and theoretical knowledge in an area of expertise. Competence level 2
includes complex practical tasks requiring a large amount of knowledge
in an area of expertise. Competence level 3 includes practical tasks
like retail, care, data processing and administration, operation of
machines and eletronical equipment, security service or
driving. Finally competence level 4 includes simple manual tasks. I
define competence levels 1 and 2 as high skill, competence levels 3
and 4 as low skill. A comprehensive list of all occupation groups as
defined by ISCO-08 and their classification into competence levels can
be found in the appendix, Tables \ref{tab:occupations_high} and
\ref{tab:occupations_low}.

By choosing occupations to define the skill level of a worker I
diverge from the standard economic literature, where agents reveal
their own skill level through their education choice. The underlying
microeconomic assumption in the literature is that obtaining a college
education is too costly for a low skill agent, such that they are
better off with a less costly education and a lower wage (signaling
theory, see \cite{Spence.1973}). However, in the case of Switzerland
it is not straight forward to differentiate between a costly and a
cheap education. Vocational educations differ widely in terms of skill
and time required and many workers complete additional, sometimes very
time consuming courses after finishing their vocational education,
which may or may not be recognized as a tertiary degree. Additionally,
obtaining a college degree in Switzerland is not very costly from a
financial point of view, as it is strongly subsidized. The only
significant cost is the opportunity cost of not being able to work
full time during the college education. For all of these reasons I
argue that a signalling model does not capture the spirit of the Swiss
solution to information assymetry in the labor market. Thus, I choose
to not differentiate workers based on their education choices, but
based on their actual occupations. This shifts the revealing mechanism
of skill levels to the principal, implying that in this paper the
underlying assumption is based on screening as opposed to signaling
(see \cite{Rothschild.1976} and \cite{Wilson.1977}).

The idea of using the ratios of wages and labor supply to estimate
elasticities of substitution between high and low skill labor
originates from \cite{Katz.1992} (hereafter, KM). In their seminal
paper, KM find an elasticity of substituion between college and high
school equivalents in the United States of around
1.4. \cite{Card.2001} (hereafter, CL) build on the KM model but drop
the assumption that, within skill groups, workers of different age
groups are perfect substitutes. They find that elasticities of
substitution between age groups are significant and in the range of
4-6 in the United States, Canada and the United
Kingdom. \cite{Blankenau.2011} (hereafter, BC) extend the KM model to
allow for different elasticities of substitution in each
industry. They identify the model using an IV-approach and find
significant differences between industries.  \cite{Fitzenberger.2006}
adress the issue of the dual education system for the case of Germany
by introducing a third skill group. Their framework distinguishes
between high skill workers (university degree), medium skill workers
(vocational training degree) and low skill workers (neither university
nor vocational training degree). Using occupation groups as a skill
definition is not well explored in the
literature. \cite{Orrenius.2007} use occupations in an investigation
on the effect of immigration on wages. They distinguish three skill
groups: professional, service-related and manual laborers. This
differentiation interesting, however it is clearly tailored to
evaluate immigration-based effects and doesn't translate well to more
general applications. As to my knowledge, \cite{Orrenius.2007} is the
only application of occupation groups as a skill identifier in a KM
model.

In this paper I argue that using occupation as a proxy for skill level
describes a labor market with a dual education system much better than
using an education-based skill classifier. I evaluate this claim by
estimating a KM-style model with Swiss data, first using education and
second using occupation as an identifier for skill level. My estimates
for the elasticities of substitution confirm this hypothesis. Using
education as a skill identifier I find no statistically significant
elasticities, implying a true elasticity of either infinity (perfect
substitutes) or an incompatibility with the theoretical model
used. Using occupations as a classifier however, I find a
statistically significant elasticity of substitution of around 2,
which is in line with previous results from the United States. This
result holds in a classic KM model and qualitatively does not change
in a CL or BC extension.

\section{Theoretical Framework}
\label{sec:theory}

\subsection{Basic Model}

The following model is based on KM, however I allow for heterogeneity
among sectors as in BC.  I start from some production function
$Y_{i,t}=f(\cdot,L_{i,t})$, where the functional form of $f()$ and its
inputs (other than labor $L_{i,t}$) are assumed to conform to standard
regularity assumption but can otherwise be chosen freely. I assume
that the labor input $L_{i,t}$ is a CES aggregate of two types of
labor (high skill $S$ and low skill $U$):
\begin{align}
  \label{eq:ces}
  L_{t}=\left( \theta_{u,i,t}U_{i,t}^{\rho_{i}} + \theta_{s,i,t}S_{i,t}^{\rho_{i}} \right)^{\frac{1}{\rho_{i}}},
\end{align}
with efficiency parameters $\theta_{u,i,t}$ and $\theta_{s,i,t}$ as
well as CES aggregator $\rho_{i}$. Under the assumption of profit
maximization and competitive markets I can then express the wage of
low-skill workers as their marginal product:
\begin{align}
  \label{eq:wage1}
  w_{u,i,t}&=\frac{\partial Y_{i,t}}{\partial U_{i,t}}=\frac{d
             Y_{i,t}}{d L_{i,t}}\frac{d L_{i,t}}{d U_{i,t}} \\
           & = \frac{d Y_{i,t}}{d L_{i,t}} \left(
             \theta_{u,i,t}U_{i,t}^{\rho_{i}} + \theta_{s,i,t}S_{i,t}^{\rho_{i}}
             \right)^{\frac{1}{\rho_{i}}-1}  \theta_{u,i,t}
             U_{i,t}^{\rho_{i}-1}.
\end{align}
Similarly, the wage of high skill workers is given as:
\begin{align}
  \label{eq:wage2}
  w_{s,i,t} = \frac{d Y_{i,t}}{d L_{i,t}} \left(
  \theta_{u,i,t}U_{i,t}^{\rho_{i}} + \theta_{s,i,t}S_{i,t}^{\rho_{i}}
  \right)^{\frac{1}{\rho_{i}}-1} 
  \theta_{s,i,t}S_{i,t}^{\rho_{i}-1},
\end{align}
and the ratio between the two wage rates is:
\begin{align}
  \label{eq:wageratio}
  \frac{w_{s,i,t}}{w_{u,i,t}}&= \frac{\theta_{s,i,t}}{\theta_{u,i,t}}\left( \frac{S_{i,t}}{U_{i,t}} \right)^{\rho_{i}-1}.
\end{align}
Now, let $\widetilde{w_{i,t}}\equiv \frac{w_{s,i,t}}{w_{u,i,t}}$,
$\widetilde{\theta_{i,t}}\equiv
\frac{\theta_{s,i,t}}{\theta_{u,i,t}}$, and
$\widetilde{s_{i,t}}\equiv \frac{S_{i,t}}{U_{i,t}}$. Taking logs and
plugging in the definition of the elasticity of substitution between
skill levels $\sigma_{S,i}=\frac{1}{1-\rho_{i}}$ yields:
\begin{align}
  \label{eq:esteq1}
  \log(\widetilde{w_{i,t}}) = \log(\widetilde{\theta_{i,t}}) - \frac{1}{\sigma_{S,i}}\log(\widetilde{s_{i,t}}).
\end{align}

\subsection{Imperfect Substitutability between Age Groups}

CL extend the KM model by dropping the assumption that, within skill
groups, workers of different age groups $j$ are perfect
substitutes. Following their approach, I define the skill-specific
labor supplies to be CES aggregates of age groups:
\begin{equation}
  \label{eq:2}
  U_{i,t}=\left[ \sum_{j}(\alpha_{u,i,j}U_{i,j,t}^{\eta_{i}}) \right]^{\frac{1}{\eta_{i}}},
\end{equation}
and
\begin{equation}
  \label{eq:3}
  S_{i,t}=\left[ \sum_{j}(\alpha_{s,i,j}S_{i,j,t}^{\eta_{i}}) \right]^{\frac{1}{\eta_{i}}},
\end{equation}
where $\alpha_{u,i,j}$ and $\alpha_{s,i,j}$ are the relative
efficiency parameters for the age groups in each industry for low
skilled workers and high skilled workers respectively and
$\sigma_{A,i}=\frac{1}{1-\eta_{i}}$ is the elasticity of substitution
between workers of different ages with the same skill level in each
industry.

The age group-specific wage rates of low-skill workers are now given
by:
\begin{align}
  w_{u,i,j,t}&= \theta_{i,l,t}U_{i,t}^{\rho_{i}-1}\Psi_{i,t} \alpha_{u,i,j}U_{i,j,t}^{\eta_{i}-1}L_{i,t}^{1-\eta_{i}} \\
             &=
               \theta_{i,l,t}U_{t}^{\rho_{i}-\eta_{i}}\Psi_{i,t}\alpha_{u,i,j}U_{i,j,t}^{\eta_{i}-1},   \label{eq:5}
\end{align}
where
\begin{equation}
  \label{eq:6}
  \Psi_{i,t}=\left( \theta_{i,l,t}U_{i,t}^{\rho_{i}}+\theta_{i,h,t}S_{i,t}^{\rho_{i}} \right)^{\frac{1}{\rho_{i}-1}}.
\end{equation}
Mirroring this result, high skill workers' wages are given by:
\begin{equation}
  \label{eq:7}
  w_{s,i,j,t}=\theta_{i,h,t}S_{i,t}^{\rho_{i}-\eta_{i}}\Psi_{i,t} \alpha_{u,i,j}S_{i,j,t}^{\eta_{i}-1}.
\end{equation}
Thus, the wage ratio will be:
\begin{equation}
  \label{eq:8}
  \frac{w_{s,i,j,t}}{w_{u,i,j,t}}=\frac{\theta_{i,h,t}}{\theta_{i,l,t}} \left( \frac{S_{i,t}}{U_{i,t}} \right)^{\rho_{i}-\eta_{i}}\frac{\alpha_{u,i,j}}{\alpha_{s,i,j}}\left( \frac{H_{i,j,t}}{L_{i,j,t}} \right)^{\eta_{i-1}}.
\end{equation}
Taking logs, substituting in $\sigma_{A,i}$ and $\sigma_{S,i}$ and
adopting the tilde-notation from the previous section leads to a model
for the observed wage gap between high and low skill workers:
\begin{align}
  \label{eq:9}
  \log \left( \widetilde{w_{i,j,t}} \right) = \log \left(
  \widetilde{\theta_{i,t}} \right) &+ \log \left(
                                     \widetilde{\alpha_{i,j}} \right) + \left[ \left(
                                     \frac{1}{\sigma_{A,i}} \right)-\left( \frac{1}{\sigma_{S,i}}
                                     \right) \right] \log \left(\widetilde{s_{i,t}} \right)
                                     \nonumber \\ &-
                                                    \left( \frac{1}{\sigma_{A,i}} \right)\log \left(
                                                    \widetilde{s_{i,j,t}} \right) + e_{i,j,t}
\end{align}
This model can be easily rearranged into a slightly more convenient
form that allows us to estimate the elasticities directly:
\begin{align}
  \label{eq:10}
  \log \left( \widetilde{w_{i,j,t}} \right) = \log \left(
  \widetilde{\theta_{i,t}} \right) &+ \log \left(
                                     \widetilde{\alpha_{i,j}} \right) +  \left(
                                     \frac{1}{\sigma_{S,i}} \right) \log \left(\widetilde{s_{i,t}} \right)
                                     \nonumber \\ &-
                                                    \left( \frac{1}{\sigma_{A,i}} \right) \left[ \log \left(\widetilde{s_{i,j,t}} \right)- \log \left( \widetilde{s_{i,t}} \right) \right] + e_{i,j,t}
\end{align}
Note that by assuming perfect substitutability between age groups, the
model collapses to the basic KM model presented in the previous
section.

\section{Estimation}
Estimation of the basic model is straight forward. First, one has to
assume some functional form for $\log(\widetilde{\theta_{i,t}})$. This
is generally assumed to be a constant with a linear trend. While this
assumption is not necessary, it makes the model simple and, as noted
by BC, provides a convenient interpretation for the coefficient of the
linear trend term in the form of the rate of skill-biased technical
change. Especially in the comparison between industries this could be
interesting. Given this assumption, equation \eqref{eq:esteq1} can be
estimated using OLS as follows:
\begin{align}
  \label{eq:est1}
  log(\tilde{w_{t}})=\beta_{0}+\beta_{1}t+\beta_{2}log(\tilde{s_{t}})+\varepsilon_{t},
\end{align}
where, as noted above, $\beta_{1}$ provides an estimate for the rate
of skill-biased technical change and $\beta_{2}$ is an estimator for
$-\frac{1}{\sigma_{S,i}}$. In order for the exogeneity assumption to
hold we need to confirm that, within a period, wage ratios are indeed
driven by skill shares and not the other way around, i.e. the wage
ratio should not affect the skill share within a period. I argue that
in the aggregate economy this is a plausible assumption, since the
only way for the aggregate skill share to change is through different
education (occupation) decisions or through migration. Both of these
effects should usually take longer than one period to manifest. In the
model with heterogeneous effects across sectors this assumption does
not hold necessarily, however. This will be discussed further in the
next subsection.
  
One major problem arises when trying to estimate equation
\eqref{eq:10} from the model with imperfect substitutability between
age groups: The modelled relationship between age-group-specific and
aggregate labor supplies depends on the elasticity of substitution
between age groups, which is one of the parameters I want to
estimate. Therefore, I cannot use both types of labor supply in my
model if I want consistent estimators of both elasticities. However,
as shown by CL, it is possible to estimate the aggregate supply using
a multi-step procedure. First, I estimate $\sigma_{A,i}$ using
equation \eqref{eq:9}, without estimating $\sigma_{A,i}$. This is
achieved by recognizing that a set of age group dummies absorbs the
relative age group productivity effect
$ \log \left( \widetilde{\alpha_{i,j}} \right)$ and a set of time
dummies absorb the relative skill group productivity effect
$ \log \left( \widetilde{\theta_{i,t}} \right)$ as well as any effects
of aggregate relative supply
$\log \left( \widetilde{s_{i,t}} \right)$. Thus, identification of
$\frac{1}{\sigma_{A,i}}$ can be achieved by estimating the following
model:
\begin{equation}
  \label{eq:11}
  \log \left( \widetilde{w_{i,j,t}} \right) = J_{j} +
  T_{t} - \frac{1}{\sigma^{A}_{i}}\log\left(
    \widetilde{s_{i,j,t}} \right) + e_{i,j,t},
\end{equation}
where $J_{j}$ and $T_{t}$ are age and year fixed effects,
respectively. With the estimates $\widehat{\frac{1}{\sigma_{i}^{A}}}$
we can then estimate $\widehat{\alpha_{u,i,j}}$ and
$\widehat{\alpha_{s,i,j}}$ by estimating the following models, derived
from equations \eqref{eq:5} and \eqref{eq:7} respectively:
\begin{equation}
  \label{eq:12}
  \log (w_{u,j,i,t}) + \widehat{\frac{1}{\sigma_{A,i}}}U_{i,j,t} = \log
  \left( \theta_{u,i,t}U_{i,t}^{\rho_{i}-\eta_{i}}\Psi_{i,t} \right) + \log(\alpha_{u,i,j})
\end{equation}

\begin{equation}
  \label{eq:13}
  \log (w_{s,j,i,t}) + \widehat{\frac{1}{\sigma_{A,i}}}S_{i,j,t} = \log
  \left( \theta_{s,i,t}S_{i,t}^{\rho_{i}-\eta_{i}}\Psi_{i,t} \right) + \log(\alpha_{s,i,j})
\end{equation}
The left hand side of these equations is observed or estimated in the
previous step. The first term of the right hand side can be absorbed
by a set of time dummies. Thus, we can estimate
$\widehat{\alpha_{u,i,j}}$ and $\widehat{\alpha_{s,i,j}}$ as the
coefficients of age group dummies in a regression of the left hand
side terms on year and age group fixed effects. Given estimates for
$\widehat{\alpha_{u,i,j}}$, $\widehat{\alpha_{s,i,j}}$ and
$\widehat{\eta_{i}}$, I return to equations \eqref{eq:2} and
\eqref{eq:3} to calculate the estimated aggregate supplies
$\widehat{U_{i,t}}$ and $\widehat{S_{i,t}}$ which I can then use to
estimate equation \eqref{eq:10} for every industry:
\begin{align}
  \label{eq:1}
  \log(\widetilde{w_{i,j,t}}) = \beta_{0,i} + \beta_{1,i}T_{t} &+ \beta_{2,i}J_{j} +
                                                                 \beta_{3,i}\log\left( \widetilde{\widehat{s_{i,t}}}\right)
                                                                 \nonumber \\ &+
                                                                                \beta_{4,i}\left[ \log\left(\widetilde{s_{i,j,t}}\right)-\log\left(\widetilde{\widehat{s_{i,t}}}\right)\right]
                                                                                + \varepsilon_{i,j,t},
\end{align}
from which I can calculate $\sigma_{S,i}=-\frac{1}{\beta_{3,i}}$ and
$\sigma_{A,i}=-\frac{1}{\beta_{4,i}}$. Additionally, $\beta_{1,i}$ can
be interpreted as the rate of skill-biased technological change.

\subsection{IV Approach}
However, as mentioned above there is one more problem remaining: While
this procedure works very well for the aggregated economy of a
country, a possible endogeneity arises when estimating it for a single
industry. It is reasonabe to assume that in the whole economy wage
differentials react to changes in skill differentials but not the
other way around, because the skill composition of the economy takes
some time to change through education or international migration. The
same is true for the age composition, which can only change very
slowly through cohort effects or international migration. This is not
necessarily true however for single industries within a country. If
labor is at least partially mobile between sectors it is plausible
that the skill and age composition of labor supply reacts to wage
changes within the same period, leading to an endogeneity issue. To
solve this issue, BC suggest an instrumental variable approach, using
the aggregate skill ratio as an instrument for the industry skill
ratio. As already discussed it is reasonable to assume that the
aggregate skill ratio is fixed within a period, however by
construction it is clearly correlated to the industry skill ratio
. The same reasoning can be applied for industry and aggregate age
specific ratios. Therefore, I use an instrumental variable approach
for the following estimations: In equation \eqref{eq:11} I use
$\frac{H_{j,t}}{L_{j,t}}$ as an instrument for
$\frac{H_{i,j,t}}{L_{i,j,t}}$. In equation \eqref{eq:1} I use
$\frac{\widehat{H_{t}}}{\widehat{L_{t}}}$ as an instrument for
$\frac{\widehat{H_{i,t}}}{\widehat{L_{i,t}}}$ and
$\frac{\frac{H_{j,t}}{L_{j,t}}}{\frac{\widehat{H_{t}}}{\widehat{L_{t}}}}$
as an instrument for
$\frac{\frac{H_{i,j,t}}{L_{i,j,t}}}{\frac{\widehat{H_{i,t}}}{\widehat{L_{i,t}}}}$. In
both cases I estimate a two stage least squares (2SLS)
model. \textcolor{red}{\emph{[to do: excess kurtosis analysis]}}
\section{Data}
\subsection{Switzerland} \label{sec:ch} All estimations for
Switzerland are based on the Swiss Labor Force Survey (SLFS),
conducted yearly (quarterly since 2010) by the Federal Statistical
Office of Switzerland. I define a time period as spanning two years,
in order to increase sample size within a period. The first time
period covers 1992-1993, the last time period 2016-2017. I define
seven five-year age groups, from 26-30 year-olds to 56-60
year-olds. For the set of estimations using education as an identifier
for skill-level, I define individuals whose highest completed
educations are mandatory school, language stays, additional
general-education schools or an apprenticeship as low skill. I define
individuals who completed a university degree (including technical and
pedagogical colleges) as high skill. I omit individuals that completed
an education in addition to an apprenticeship, higher vocational
education or the ``Gymnasium'' but did not complete a university
degree. For the calculation of the size of the labor force $H_{i,j,t}$
and $L_{i,j,t}$ I sum hours worked of all workers between the age of
15 and 65 by sex, age group, year and skill level. I then divide hours
worked of the high skill group by hours worked of the low skill group,
omitting cases with 5 or less observations.  To calculate the wage
gap, I remove observations with an hourly wage above 500 Swiss Francs
or below 5 Swiss Francs. I only consider workers that are employed
full time (ignoring part-time workers and self-employed). To estimate
the wage gap, I separately regress hourly log wages on a skill level
dummy, a gender dummy\footnote{Typically, in the literature the wage
  gap is estimated separately for men and women or exclusively for
  men. I do not do this due to sample size issues in the
  industry-specific estimations. Certain combinations of age, industry
  and gender yield very low observations, in some cases even 0
  observations. However, the results for the aggregate economy do not
  change in any significant way depending on which specification I
  use.} and a linear age term in every time period, age group and
industry. In the model without age-specific effects I do not split the
sample into age groups but include a squared age term. As in CL, the
inverse of the estimated variance of the coefficient for skill level
is later used as weight in the models.
\subsection{United States}
\label{data_us}
Estimations for the United States are based on the March CPS Extracts
from 1980-2016 \citep{CenterforEconomicandPolicyResearch.2016}. Due to
data irregularities (redefinition of occupation classification) in the
first two years, I only use data from 1982-2016, again pooled in two
year periods. For the set of estimations using education as an
identifier for skill-level, I define individuals with exactly a
Highschool degree as low-skill and individuals with exactly a college
degree as high-skill. I omit individuals with less than a Highschool
degree, those with ``some college'' and those with an ``advanced''
degree. Calculation of the size of the labor force is identical is
with the Swiss data (see \ref{sec:ch}). Calculation of the wage gap is
based on weekly wages, where weekly wages below USD 50 are omitted. I
only consider full time employed individuals. Estimation of the wage
gap is almost identical as with Swiss data, however, I also control
for race by including a ``non-white'' dummy.
\subsection{Coding Issues}
\label{issues}
One major problem arises when attempting to replicate the distinction
between Competence Levels in data from the United States. The
Competence Levels are based on the ISCO-08 classification which is
commonly used in European Data, however it is not used in the United
States. The CPS uses its own classification for occupations which is
based on the US census. As I am not aware of a classification that is
similar to the Swiss Competence Levels in US data, I recode the CPS
occupations into ISCO-08 occupations using crosswalks from the Bureau
of Labor Statistics. However, this does not always lead to unique
correspondences, i.e. a CPS occupation group might be identified as
several ISCO occupation groups which can lead to conflicts when
deciding on its Competence Level. More problematic is the fact that
the CPS occupation classification has undergone several revisions
during my observation period. Every revision requires an additional
step of recoding which can lead to significant disturbances. If
aggregate skill shares and wage ratios both change due to a revision,
this makes identification of the elasticity impossible. We cannot
disentangle the effect of the revision from the effect of actual skill
changes in the economy. In order to correctly identify the effect of
occupational changes in the United States, I would need a different
classification of skill levels. However, in the absence of such a
classification I still present the results of the model using
Competence Levels, under the caveat that they are flawed and should
not be given much weight.

\newpage
\section{Descriptive Evidence}
First, I present graphical evidence on the relationship between
relative employment and relative wages. Figure \ref{fig:shares1} shows
these shares in Switzerland with the education-based skill
classifier. The wage gap seems to be very constant over time, whereas
the skill share has been increasing immensely since 2000. As discussed
in Section \ref{intro}, this can most likely be attributed to the
creation of the technical colleges and the accompanying redefinition
of high-skilled workers. Since this is not an actual change in the
underlying skill level of the workers, it is not surprising that this
has not lead to a corresponding decrease in the wage gap.
\begin{figure}[H]
  \centering
  \includegraphics[width=0.9\textwidth]{Z:/OLG_CGE_Model/code/elasticities/7inds/aggr_shares_coll.png}
  \caption{Skill Shares and Wage Ratios - Education - CH}
  \label{fig:shares1}
\end{figure}
Figure \ref{fig:shares2} shows the same data but with occupation as
identifier for skill. Here, the data looks to be much more in line
with my hypothesis. The inverse relationship between wage gaps and
skill gaps is not perfect, but clearly visible in several time
periods.
\begin{figure}[h]
  \centering
  \includegraphics[width=0.9\textwidth]{Z:/OLG_CGE_Model/code/elasticities/7inds/aggr_shares_skill.png}
  \caption{Skill Shares and Wage Ratios - Occupation - CH}
  \label{fig:shares2}
\end{figure}
Changing focus to the United States, Figure \ref{fig:shares3} shows
the skill shares and wage ratios between High School and College
graduates. While both ratios are clearly trending upwards, one can
still observe some negative relationship between the two, however this
is hard to see from the graph.
\begin{figure}[h]
  \centering
  \includegraphics[width=0.9\textwidth]{Z:/OLG_CGE_Model/code/elasticities/7inds/aggr_shares_coll_us.png}
  \caption{Skill Shares and Wage Ratios - Education - US}
  \label{fig:shares3}
\end{figure}
Finally, using the Competence Level-classification of occupations,
Figure \ref{fig:shares4} reveals the coding issues mentioned in
section \ref{issues}. The stark drop of the skill share in 1982 can
only be attributed to a change in the definition of occupations in the
US data and an associated change in the way these occupations are
translated to ISCO-08 and finally to competence levels. This issue is
solved by only considering data from 1982 onward. A similar change
happened in 2004. This is visible in the way both skill share and wage
gap move in the same direction and probably one of the reasons for my
conflicting results.
\begin{figure}[h]
  \centering
  \includegraphics[width=0.9\textwidth]{Z:/OLG_CGE_Model/code/elasticities/7inds/aggr_shares_skill_us.png}
  \caption{Skill Shares and Wage Ratios - Occupation - US}
  \label{fig:shares4}
\end{figure}
Figures \ref{fig:shares_ind1} - \ref{fig:shares_ind4} in the Appendix
show the same graphs disaggregated by industry.


Table \ref{tab:mapping} shows the share of education levels of the
different competence levels. There is a clear correlation, however it
is not as strong as one might expect.
% Table generated by Excel2LaTeX from sheet 'Sheet1'
\begin{table}[H]
  \centering
  \caption{Share of education levels in each competence level,
    2016-2017}
  \begin{tabular}{lllll}
    \toprule
    \multicolumn{1}{p{5.5em}}{Competence Level} & \multicolumn{1}{p{6.3em}}{No Vocational Education} & \multicolumn{1}{p{5.8em}}{Vocational Education} & \multicolumn{1}{p{4.5em}}{Higher Education} & \multicolumn{1}{p{5em}}{University} \\
    \midrule
    1     & 2.0   & 13.9  & 23.0  & 61.2 \\
    2     & 4.2   & 39.2  & 35.1  & 21.4 \\
    3     & 18.4  & 55.9  & 19.7  & 6.0 \\
    4     & 46.2  & 38.3  & 12.5  & 3.1 \\
    \bottomrule
  \end{tabular}%
  \label{tab:mapping}%
\end{table}%
\textcolor{red}{[some more descriptives to be added]}
\section{Results}
\subsection{Basic Model}
First, I present the results for the KM model (aggregate economy,
perfect substitution over age groups). The left-hand side of Table
\ref{noage_main} lists the coefficients and standard
errors\footnote{Due to the multi-step estimation procedure of the CL
  model, all standard errors are based on a bootstrap over the
  disaggregated sample with 1000 replications. As a robustness check I
  present the results of the KM model with analytical standard errors
  in Section \ref{robustness}} for Switzerland, where the first column
uses education as a proxy for skill level and the second column uses
occupation as a proxy for skill level. In this simple model, I do not
find a statistically significant elasticity of substitution between
workers with a university degree and those with an
apprenticeship. This could mean that either these groups are perfect
substitutes or that the classification does not approximate skill
accurately. The estimated elasticity of substitution between workers
with an occupation classified as high skill and those with an
occupation classified as low skill is strikingly different,
however. Here I estimate a statistically significant elasticity of
2.037 (which corresponds to an estimated coefficient of roughly
-0.49). Additionally I estimate the rate of skill-biased technical
change to be roughly 3.5\% over a two year period.

Repeating the estimation with US data yields very different
results. The estimated elasticity of substitution between college and
high school educated workers is statistically significant and around
1.996 (corresponding to a coefficient of roughly -0.50).  As a point
of reference, using the same model but older data, \cite{Katz.1992}
find an elasticity of 1.41, \cite{Card.2001} find 2.41 (on a sample of
only males, the estimate for both genders is lower) and
\cite{Blankenau.2011} find 1.4. Using a different method,
\cite{PerKrusell.2000} find an elasticity of 1.67. Thus, my estimates
are slightly higher than the usual findings, however, I do use more
recent data. Looking at the last column, the elasticity of
substitution between occupations classified as high skill and those
classified as low skill in the United States is estimated to be
negative and statistically significant.  As this is a theoretical
impossibility, it directly invalidates the underlying model. Since the
same model is valid with an education-based skill definition, this
finding suggests that the way I defined the occupation-based skill
groups is not suitable for the American labor market. There are two
reasons as to why this might be the case. Firstly, it might be that
occupation definitions are simply not very important in the American
labor market and should therefore not be used to approximate skill
levels. Secondly, it is possible that the Swiss definition of
Competence Levels does not translate to the American occupation
definition. This effect is reinforced by the coding issues mentioned
in section \ref{data_us}.

\begin{table}[H]
  \centering
  \caption{Aggregate Economy Results - Perfect Substitution over Age
    Groups}
  \input{noage/main_ch.tex}
  \label{noage_main}
\end{table}%

\subsection{Imperfect Substitutability between Age Groups}
Allowing for imperfect substitution between different age groups
within skill groups decreases the standard error of the education
coefficient in Switzerland which further invalidates this definition
of skill in Switzerland. It also increases the estimated elasticities
in both Switzerland and the United States, implying that a part of the
effect measured in the simple model can be captured by age
effects. Indeed, the elasticities of substitution between age groups
is statistically significant in Switzerland as well as the United
States. Again, the estimates of 7.2 and 6.6, respectively, are on the
higher end of previous results, but still plausible (CL find estimates
between 4-6 for the US, UK and Canada).

\begin{table}[H]
  \centering
  \caption{Aggregate Economy Results - Imperfect Substitution over Age
    Groups}
  \input{7inds/main_ch.tex}
  \label{age_main}
\end{table}%

\subsection{Industry-Specific Results}
\textcolor{red}{[IV: preliminary results, subject to change! First
  stage results to be added.]}  Tables \ref{indresults_noage} and
\ref{indresults_age_skill_ch} show the industry-specific estimates of the
elasticities of substitution between skill levels for the US and
Switzerland, without and with age effects. The industry-specific
results for the USA (Education) are fairly imprecise. While the point
estimates mostly are reasonable, the standard errors are fairly large
leading to only few industries returning statistically significant
results. The results for Switzerland (Education) are more
precise. Only Construction and Health return statistically clearly
insignificant estimates. Especially striking are the results for the
Finance sector, for which the estimates are rather low and not
significantly different from 1 (Cobb-Douglas case). This result is
significant and robust over different specifications. This result is
special insofar as it implies that high and low skill labor might be
net complements, as opposed to net substitutes like in all other
industries. (for the age and time trend results refer to tables
\ref{indresults_trend_noage}, \ref{indresults_age} and
\ref{indresults_trend_age} in the appendix)

\begin{table}[H]
  \centering
  \caption{Industry Results - No Age Effects - Skill}
  \input{noage/ind_skill_ch.tex}
  \label{indresults_noage}
\end{table}%

\begin{table}[H]
  \centering
  \caption{Industry Results - Age Effects - Skill}
  \input{7inds/ind_skill_ch.tex}
  \label{indresults_age_skill_ch}
\end{table}%

\begin{table}[H]
  \centering
  \caption{Industry Results - Age Effects - Skill}
  \input{7inds/ind_age_ch.tex}
  \label{indresults_age_ch}
\end{table}%



\subsection{Robustness Analyses}
\label{robustness}
\textcolor{red}{Not finished yet}

\section{Conclusion}
From the results it is almost surprisingly clear, that the standard
definition of skill levels is not suitable for countries with a dual
education system like Switzerland. On the other hand, using the
suggested occupation-based skill classification is not applicable to
the United States. There are two different potential explanations for
this observation. First, it is possible that the way education-based
and occupation-based skill levels are defined is chosen wrong and
doesn't translate well between the systems. For example, in some
occupations it might be the case that a suitable signal for being
high-skilled is additional vocational training, whereas a university
degree might be considered too theoretical and does not signal high
skill for a more practical task. Distinguishing these cases however is
not straight-forward without further research in the area. One reason
for the fact that the competence level distinction is so clearly
invalidated in the US is that the recoding from the occupation
classification used in the US data (CPS occupational titles) to the
European ISCO is very cumbersome and sometimes impossible to do
correctly. For example, according to the official crosswalks, the
occupation ``Veterinary Assistants and Laboratory Animal Caretakers''
in the US CPS data corresponds to either ``Veterinary technicians and
assistants'' (high skill) or ``Pet groomers and animal care workers ''
(low skill) in the ISCO. It is not possible to make the distinction
between low and high skill workers in this case. There are many such
cases which could lead to significant distortions. A different
explanation for my findings is of theoretical nature. It is possible
that the fundamental skill revealing mechanism is different between
the United States and Switzerland, meaning that in the United States
signalling is prevalent (agents choose to attend college based on
their underlying skill level) whereas in Switzerland screening is
prevalent (high skill workers self-select into occupations that
include skill-intensive tasks). There is some anecdotal evidence that
supports this claim, but further research in this area is required to
make any conclusions.

Whatever the underlying reason for my results is, it should be clear
that any models of the Swiss economy with disaggregation on skill
levels should pay attention to the definition of skill level. Blindly
adopting definitions from US models could lead to misleading results
or inconsistencies. I do not claim that competence levels are the best
possible approximation for skill levels. There may be other suitable
approximations that might perform better. One interesting example
would be a task-based approach such as in, for example,
\cite{SpitzOener.2006}. More research is also necessary in terms of
classifying occupations in the United States correctly. A more
realistic classification could solve my conflicting results, allow for
better international comparisons and might even shed light on the
underlying skill revealing mechanisms of different labor markets.


\newpage
\bibliography{references}
\appendix
\section{Appendix}
% Table generated by Excel2LaTeX from sheet 'Sheet1'
\begin{table}[H]
  \centering
  \caption{High Skill Occupations}
  \begin{tabular}{rrp{18em}}
    \toprule
    \multicolumn{1}{l}{Competence Level} & \multicolumn{1}{l}{ISCO-08 Code} & \multicolumn{1}{l}{Occupation} \\
    \midrule
    1     & 1     & Commissioned armed forces officers \\
                                         & 10    & Managers, nos \\
                                         & 11    & Chief Executives, Senior Officials and Legislators \\
                                         & 12    & Administrative and Commercial Managers \\
                                         & 13    & Production and Specialized Services Managers \\
                                         & 14    & Hospitality, retail and other services managers \\
                                         & 20    & Professionals, nos \\
                                         & 21    & Science and engineering professionals \\
                                         & 22    & Health professionals \\
                                         & 23    & Teaching professionals \\
                                         & 24    & Business and administration professionals \\
                                         & 25    & Information and communications technology professionals \\
                                         & 26    & Legal, social and cultural professionals \\
    \midrule
    2     & 30    & Technicians and associate professionals, nos \\
                                         & 31    & Science and engineering associate professionals \\
                                         & 32    & Health associate professionals \\
                                         & 33    & Business and administration associate professionals \\
                                         & 34    & Legal, social, cultural and related associate professionals \\
                                         & 35    & Information and communications technicians \\
    \bottomrule
  \end{tabular}%
  \label{tab:occupations_high}%
\end{table}%

% Table generated by Excel2LaTeX from sheet 'Sheet1'
\begin{table}[H]
  \centering
  \caption{Low Skill Occupations}
  \begin{tabular}{rrp{20em}}
    \toprule
    \multicolumn{1}{l}{Competence Level} & \multicolumn{1}{l}{ISCO-08 Code} & \multicolumn{1}{l}{Occupation} \\
    \midrule
    3     & 2     & Non-commissioned armed forces officers \\
                                         & 40    & Clerical support workers, nos \\
                                         & 41    & General and keyboard clerks \\
                                         & 42    & Customer services clerks \\
                                         & 43    & Numerical and material recording clerks \\
                                         & 44    & Other clerical support workers \\
                                         & 50    & Service and sales workers, nos \\
                                         & 51    & Personal service workers \\
                                         & 52    & Sales workers \\
                                         & 53    & Personal care workers \\
                                         & 54    & Protective services workers \\
                                         & 60    & Skilled agricultural, forestry and fishery workers, nos \\
                                         & 61    & Market-oriented skilled agricultural workers \\
                                         & 62    & Market-oriented skilled forestry, fishery and hunting workers \\
                                         & 63    & Subsistence farmers, fishers, hunters and gatherers \\
                                         & 70    & Craft and related trades workers, nos \\
                                         & 71    & Building and related trades workers, excluding electricians \\
                                         & 72    & Metal, machinery and related trades workers \\
                                         & 73    & Handicraft and printing workers \\
                                         & 74    & Electrical and electronic trades workers \\
                                         & 75    & Food processing, wood working, garment and other craft and related trades workers \\
                                         & 80    & Plant and machine operators and assemblers, nos \\
                                         & 81    & Stationary plant and machine operators \\
                                         & 82    & Assemblers \\
                                         & 83    & Drivers and mobile plant operators \\
    \midrule
    4     & 3     & Armed forces occupations, other ranks \\
                                         & 90    & Elementary occupations, nos \\
                                         & 91    & Cleaners and helpers \\
                                         & 92    & Agricultural, forestry and fishery labourers \\
                                         & 93    & Labourers in mining, construction, manufacturing and transport \\
                                         & 94    & Food preparation assistants \\
                                         & 95    & Street and related sales and service workers \\
                                         & 96    & Refuse workers and other elementary workers \\
    \bottomrule
  \end{tabular}%
  \label{tab:occupations_low}%
\end{table}%


\begin{figure}[H]
  \centering
  \includegraphics[width=0.9\textwidth]{gap_college.png}
  \caption{Log-Wages and Wage Gap of College vs. High School}
  \label{fig:gap_college}
\end{figure}

\begin{figure}[H]
  \centering
  \includegraphics[width=0.9\textwidth]{gap_skills.png}
  \caption{Log-Wages and Wage Gap of Competence Levels}
  \label{fig:gap_skills}
\end{figure}

\begin{figure}[H]
  \centering
  \includegraphics[width=0.9\textwidth]{agegroups.png}
  \caption{Age Profiles of Wage Gaps}
  \label{fig:gap_age}
\end{figure}

\begin{figure}[H]
  \centering
  \includegraphics[width=0.9\textwidth]{labor_aggregate.png}
  \caption{Aggregate Relative Employment}
  \label{fig:labor_aggregate}
\end{figure}

\begin{figure}[H]
  \centering
  \includegraphics[width=0.9\textwidth]{labor_ind.png}
  \caption{Relative Employment by Industry}
  \label{fig:labor_ind}
\end{figure}

\begin{figure}[H]
  \centering
  \includegraphics[width=0.9\textwidth]{labor_age_ind.png}
  \caption{Relative Employment by Industry and Age}
  \label{fig:labor_age_ind}
\end{figure}


\begin{figure}[H]
  \centering
  \includegraphics[width=0.9\textwidth]{Z:/OLG_CGE_Model/code/elasticities/7inds/laborwage_skill.png}
  \caption{Skill Shares and Wage Ratios - Occupation - CH}
  \label{fig:shares_ind1}
\end{figure}

\begin{figure}[H]
  \centering
  \includegraphics[width=0.9\textwidth]{Z:/OLG_CGE_Model/code/elasticities/7inds/laborwage_college.png}
  \caption{Skill Shares and Wage Ratios - Education - CH}
  \label{fig:shares_ind2}
\end{figure}

\begin{figure}[H]
  \centering
  \includegraphics[width=0.9\textwidth]{Z:/OLG_CGE_Model/code/elasticities/7inds/laborwage_college_us.png}
  \caption{Skill Shares and Wage Ratios - Education - US}
  \label{fig:shares_ind3}
\end{figure}

\begin{figure}[H]
  \centering
  \includegraphics[width=0.9\textwidth]{Z:/OLG_CGE_Model/code/elasticities/7inds/laborwage_skill_us.png}
  \caption{Skill Shares and Wage Ratios - Occupation - US}
  \label{fig:shares_ind4}
\end{figure}

\begin{table}[H]
  \centering
  \caption{Industry Results - Trend - No Age Effects}
  \input{Z:/OLG_CGE_Model/code/elasticities/noage/ind_trend.tex}
   \label{indresults_trend_noage}
\end{table}%

\begin{table}[H]
  \centering
  \caption{Industry Results - Age - Age Effects}
  \input{Z:/OLG_CGE_Model/code/elasticities/7inds/ind_age.tex}
   \label{indresults_age}
\end{table}%

\begin{table}[H]
  \centering
  \caption{Industry Results - Trend - Age Effects}
  \input{Z:/OLG_CGE_Model/code/elasticities/7inds/ind_trend.tex}
  \label{indresults_trend_age}
\end{table}%

%\begin{table}[H]
%  \centering
%  \caption{Industry Results Switzerland - Trend}
%  \input{Z:/OLG_CGE_Model/code/elasticities/7inds/ind_trend_ch.tex}
%  \label{indresults_trend_ch}
%\end{table}%


\end{document}
